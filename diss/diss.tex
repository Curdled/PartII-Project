\documentclass[12pt,a4paper,twoside,openright]{book}
\usepackage[subpreambles=true]{standalone}
\usepackage[backend=biber,style=numeric,sorting=none]{biblatex}
\usepackage[pdfborder={0 0 0}]{hyperref}    % turns references into hyperlinks
\usepackage[margin=25mm]{geometry}  % adjusts page layout
\usepackage{graphicx}  % allows inclusion of PDF, PNG and JPG images
\usepackage{verbatim}
\usepackage{docmute}   % only needed to allow inclusion of proposal.tex
\usepackage[utf8]{inputenc}
\usepackage{textcomp}
\usepackage[utf8]{inputenc}
\usepackage[english]{babel}
\usepackage{import}

\usepackage{amsfonts}
\usepackage{amssymb}
\usepackage{amsthm}
\usepackage{sansmath}
\usepackage{mathtools}
\usepackage{stmaryrd}


\usepackage{pgfplots}
\pgfplotsset{compat=newest}
%% the following commands are needed for some matlab2tikz features
\usetikzlibrary{plotmarks}
\usetikzlibrary{arrows.meta}
\usepgfplotslibrary{patchplots}
\usepackage{grffile}
\usepackage{amsmath}

\usepackage{csquotes}

\usepackage{pgfgantt}
\usepackage{graphicx}
\usepackage{xcolor}


\ganttset{group/.append style={orange},
          milestone/.append style={red},
          progress label node anchor/.append style={text=red}}

\usepackage{lmodern}
\usepackage{textcomp}
\usepackage[T1]{fontenc}
\usepackage{listings}

\usepackage[utf8]{inputenc}
\usepackage[english]{babel}

\definecolor{pblue}{rgb}{0.13,0.13,1}
\definecolor{pgreen}{rgb}{0,0.5,0}
\definecolor{pred}{rgb}{0.9,0,0}
\definecolor{pgrey}{rgb}{0.46,0.45,0.48}

\usepackage{listings}
\lstnewenvironment{JavaLst}
{\lstset{language=Java,
  showspaces=false,
  showtabs=false,
  breaklines=true,
  showstringspaces=false,
  breakatwhitespace=true,
  commentstyle=\color{pgreen},
  keywordstyle=\color{pblue},
  stringstyle=\color{pred},
  basicstyle=\ttfamily,
  escapeinside={&}{&},
  moredelim=[il][\textcolor{pgrey}]{$$},
  moredelim=[is][\textcolor{pgrey}]{\%\%}{\%\%}
}}{}

\lstdefinestyle{HaskellStyle}{
  frame=none,
  xleftmargin=2pt,
  stepnumber=1,
  numbersep=5pt,
  numberstyle=\ttfamily\tiny\color[gray]{0.3},
  belowcaptionskip=\bigskipamount,
  commentstyle=\color{pgreen},
  keywordstyle=\color{pblue},
  stringstyle=\color{pred},
  basicstyle=\ttfamily,
  captionpos=b,
  escapeinside={&}{&},
  language=haskell,
  tabsize=2,
  emphstyle={\bf},
  commentstyle=\it,
  showspaces=false,
  columns=flexible,
  showstringspaces=false,
  morecomment=[l]
}

\lstnewenvironment{HaskellLst}{%
  \lstset{style=HaskellStyle}}{}

\lstnewenvironment{JVMLst}
{\lstset{
  frame=none,
  numbers=none,
  xleftmargin=2pt,
  language=JVMIS,
  emphstyle={\bf},
  basicstyle=\ttfamily,
  commentstyle=\it,
  % stringstyle=\mdseries\rmfamily,
  commentstyle=\color{pgreen},
  stringstyle=\color{pred},
  keywordstyle=\color{black},
  basicstyle=\ttfamily
  % basicstyle=\small\sffamily
}} {}


\renewcommand{\rmdefault}{bch} % change default font

\usepackage[english]{babel}
\usepackage[utf8]{inputenc}
\usepackage{tikz} 
\usepackage{forest}
\usepackage{calc}
\usetikzlibrary{arrows,decorations.pathmorphing,backgrounds,fit,positioning,shapes.symbols,chains}

% Proof 
\newtheorem{theorem}{Theorem}
\newtheorem{lemma}{Lemma}
\newtheorem{case}{Case}
\newtheorem{subcase}{Case}
\numberwithin{subcase}{case}

\newcommand{\tlang}{\bigstar}
\newcommand{\thunk}[1]{\lceil #1 \rceil}
\newcommand{\unwrap}[1]{\lfloor #1 \rfloor}
\newcommand{\tcbn}{\rightarrow_N}
\newcommand{\tcbv}{\rightarrow_V}
\newcommand{\tccbv}{\rightarrow_V^*}
\newcommand{\tthunk}{\rightarrow_\tlang}
\newcommand{\tlthunk}{\rightsquigarrow_\tlang}
% End Proof

\newlength\fwidth
\setlength{\fwidth}{\textwidth}


\raggedbottom                           % try to avoid widows and orphans
\sloppy
\clubpenalty1000%
\widowpenalty1000%

\renewcommand{\baselinestretch}{1.1}    % adjust line spacing to make

\newlength\gwidth
\newlength\gheight
\setlength{\gwidth}{\textwidth}
\setlength{\gwidth}{\textheight}

\newcommand{\importMGraph}[3]{\setlength{\gwidth}
  {#2}\setlength{\gheight}{#3}\subimport{graphs/}{#1}}
\newcommand{\namefig}{\textbf{Figure}~}
\newcommand{\nametable}{\textbf{Table}~}

\addbibresource{refs.bib}
% more readable

\usepackage{tikz}
\usetikzlibrary{positioning}

\tikzset{
  treenode/.style = {shape=rectangle, rounded corners,
                     draw, align=center,
                     top color=white, bottom color=blue!20},
  root/.style     = {treenode, font=\Large},
  env/.style      = {treenode, font=\ttfamily\normalsize},
  dummy/.style    = {circle,draw}
}

\forestset{
  make tab/.style args={#1:#2:#3/#4:#5:#6/#7:#8:#9}{%
    content={%
      \tabcolsep=.6\tabcolsep
      \begin{tabular}{p{\widthof{x}}|p{\widthof{x}}|p{\widthof{x}}}
        #1 & #2 & #3\\\hline#4&#5&#6\\\hline#7&#8&#9
      \end{tabular}}},
  label position r/.initial=right,
  label position b/.initial=below
}


\begin{document}


%%%%%%%%%%%%%%%%%%%%%%%%%%%%%%%%%%%%%%%%%%%%%%%%%%%%%%%%%%%%%%%%%%%%%%%%
% Title

\thispagestyle{empty}

\pagestyle{empty}
\begingroup

\rightline{\LARGE \textbf{Joe Isaacs}}

\vspace*{60mm}
\begin{center}
\Huge
\textbf{Haskell 98 to JVM bytecode compiler} \\[5mm]
Computer Science Tripos -- Part II \\[5mm]
Sidney Sussex College \\[5mm]
\today  % today's date
\end{center}


\newpage

\thispagestyle{empty}

%%%%%%%%%%%%%%%%%%%%%%%%%%%%%%%%%%%%%%%%%%%%%%%%%%%%%%%%%%%%%%%%%%%%%%%%%%%%%%
% Proforma, table of contents and list of figures

\pagestyle{plain}

\frontmatter

\chapter*{Proforma}

{\large
\begin{tabular}{ll}
Name:               & \bf Joe Isaacs                            \\
College:            & \bf Sidney Sussex College                 \\
Project Title:      & \bf Haskell 98 to JVM bytecode compiler   \\
Examination:        & \bf Computer Science Tripos -- Part II, July 2017  \\
Word Count:         & \bf 10980\footnotemark[1]
                    (well less than the 12000 limit)  \\
Project Originator: & Joe Isaacs                      \\
Supervisor:         & R.~Kovacsics                    \\ 
\end{tabular}
}
\footnotetext[1]{This word count was computed
by \texttt{detex diss.tex | tr -cd '0-9A-Za-z $\tt\backslash$n' | wc -w}
}
\stepcounter{footnote}


\section*{Declaration}

I, Joe Isaacs of Sidney Sussex College, being a candidate for Part II of the Computer Science Tripos, hereby declare
that this dissertation and the work described in it are my own work,
unaided except as may be specified below, and that the dissertation
does not contain material that has already been used to any substantial
extent for a comparable purpose.

\bigskip
\leftline{Signed Joe Isaacs}

\medskip
\leftline{Date \today}

\tableofcontents

\listoffigures
\begingroup
\let\clearpage\relax
\listoftables
\endgroup

\newpage

\mainmatter

\pagestyle{headings}

\chapter{Introduction}

\import{tex/introduction/}{intro.tex}

\chapter{Preparation}

\import{tex/preparation/}{prep.tex}

\chapter{Implementation}

\import{tex/implementation/}{impl.tex}

\chapter{Evaluation}

\import{tex/evaluation/}{eval.tex}

\chapter{Conclusion}

\import{tex/conclusion/}{conc.tex}


%%%%%%%%%%%%%%%%%%%%%%%%%%%%%%%%%%%%%%%%%%%%%%%%%%%%%%%%%%%%%%%%%%%%%
% the bibliography
\addcontentsline{toc}{chapter}{Bibliography}
\printbibliography

%%%%%%%%%%%%%%%%%%%%%%%%%%%%%%%%%%%%%%%%%%%%%%%%%%%%%%%%%%%%%%%%%%%%%
% the appendices
\appendix

\chapter{Byte Code Translation Document}

\label{appendix:bytecodetranslationdoc}

\import{tex/bytecodetranslation/}{bytecodetranslation.tex}

\chapter{Programs of interest}

Note that for the reader convenience the programs as displayed as standard
Haskell 98 dialect, not JVHC valid Haskell.

\section*{Fib}
\lstinputlisting[language=Haskell,style=HaskellStyle]{progs/fib.hs}
\section*{TestN}
\lstinputlisting[language=Haskell,style=HaskellStyle]{progs/testN.hs}
\section*{TiExpr}
\lstinputlisting[language=Haskell,style=HaskellStyle]{progs/tiExpr.hs}
\section*{BadProg}
\lstinputlisting[language=Haskell,style=HaskellStyle]{progs/BadInlineProg.hs}

\label{appendix:programs}

\chapter{Thunk Proof}

\import{tex/evaluation/proof/}{thunkProof.tex}

\label{appendix:thunkProof}

\chapter{Project Proposal}

\import{tex/proposal/}{proposal.tex}

\label{appendix:proposal}


\end{document}
