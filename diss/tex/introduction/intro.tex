\documentclass[float=false, crop=false]{standalone}
\usepackage[subpreambles=true]{standalone}
\usepackage{import}

\usepackage{subfiles}
\usepackage[backend=biber,style=numeric,sorting=none]{biblatex}
\usepackage[pdfborder={0 0 0}]{hyperref}    % turns references into hyperlinks
\usepackage[margin=25mm]{geometry}  % adjusts page layout
\usepackage{graphicx}  % allows inclusion of PDF, PNG and JPG images
\usepackage[utf8]{inputenc}
\usepackage{textcomp}
\usepackage{import}
\usepackage[font=small,labelfont=bf]{caption}

\usepackage{amsmath}

\usepackage{csquotes}
\usepackage{color}

\usepackage{listings}

\newlength\gwidth
\newlength\gheight
\setlength{\gwidth}{\textwidth}
\setlength{\gwidth}{\textheight}

\newcommand{\namefig}{\textbf{Figure}~}

\begin{document}
\section{Motivation}

Functional programming has a strong basis in mathematics, it is
loosely influenced by Church's Lambda calculus, this means
programs written in Haskell are very easy to reason about mathematically.
There are marked trends of programmers moving towards functional 
languages as correctness properties are usually much easier to verify.
Haskell is a functional language first appearing in the 1990s.
It is a pure, lazy functional language. By pure it is meant 
that there are no side effects when functions are run, which means
when a function is run the result will always be the same, every time 
the function is run. Lazy language means that
expressions in the language are only evaluated at the latest possible
point, meaning if an expression is never required by a function, then 
the expression will never be reduced to a value.
The project proposes to implement a Haskell to JVM compiler called 
the Java Virtual-machine Haskell Compiler (JVHC), 
this will allow the benefits of Haskell to be used on the JVM. This means
Haskell code can be run on the JVM, but moreover Haskell code
can take advantage of libraries already written for the JVM. 
The JVM is widely used in industry, and large software companies
have invested heavily into software that works on the JVM. This project
will, therefore, benefit people with large investments in the JVM, 
but also will allow the industry to benefit from Haskell for new and
existing projects.


% \begin{itemize}
%   \item 
%       Functional programming languages can be very useful in solving
%       problems, same for \textcolor{red}{imperative or 
%         object oriented languages}.
%       Having Haskell on the JVM will allow fusion between all 
%       libraries written in JVM based language and the dialect of Haskell
%       supported by JVHC.
%   \item 
%     Haskell is lazy, so can be useful in some problem domains to 
%     use a lazy language, gives more options to users of the JVM.
% \end{itemize}

\section{Overview of Haskell and its runtime}

The Haskell 98 \cite{haskell98-spec} report defines the Haskell language 
and a small standard library.
Haskell can be thought of as a lambda calculus, with an ML-like type system,
but with qualified types \cite{qualified-types}, known in 
Haskell as Type Classes. This compiler will
leave out parts of the standard library, however in the future 
iterations of the compiler these 
features could be implemented, since the features are available via the JVM.
The library is incomplete due to time constraints of the project and
also adding new functionality will not show any new skills.
The compiler will also not support Type classes. The interesting
part of the compiler project will be implementing a lazy language (Haskell)
on the JVM with eager semantics.

Haskell allows definition of infinite data structures:

\begin{lstlisting}
nat :: Int -> [Int]
nat x = x : nat (x+1)
\end{lstlisting}

This will be a list of all the integers starting from \texttt{x}, 
and going on forever. Since this list has no 
bound on the length it must be the case that values are computed
only when required, this is possible due to the lazy nature
of Haskell. The first line is an optional type signature, Haskell
uses \verb|::| to denote ``is of type'' and a single \verb|:| to denote cons (list
append), unlike ML-like languages.

% \begin{itemize}
%   \item Explanation of simple Haskell features.

%   \item Define terms, lazy, pure, and functional.

%   \item Overview of writing idiomatic Haskell.

%   \item Talk about monads, since they are used throughout the 
%     implementation of the project.
  
%   \item Find an example of code that can be implemented more 
%     elegantly in Haskell then in Java.
% \end{itemize}

\section{Related work}

There are many implementations of the Haskell 98 report.
The de-facto standard implementation is GHC \cite{ghc} (Glasgow Haskell
Compiler), this compiler generates native machine code. There
are implementations that compile Haskell to JVM bytecode, including Eta \cite{eta}
and Frege \cite{frege}. As other implementations exist I can then 
compare my implementation to these compilers in the evaluation stage
of the project. 

One of the proposed extensions would be to implement an optimization,
I chose functional inlining, however there are other optimizations, which
would work especially well for this compiler. One of these
optimizations would be strictness analysis and
then transformations that can be associated with this analysis. 
However implementing this was not possible given the timescale of the project.


% \begin{itemize}
%   \item Other implementations of Haskell on the JVM include eta and
%     Frege. Then the baseline implementation of Haskell
%     is GHC, which the code for both Frege and eta is forked from (both
%     with their custom back-ends),.

%   \item 
%     Could talk about other possible optimizations, such as strictness
%     analysis, maybe compare to with inlining and justify why
%     inlining was a good choice.
% \end{itemize}

\end{document}
