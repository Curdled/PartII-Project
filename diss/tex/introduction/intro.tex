\documentclass[float=false, crop=false]{standalone}
\usepackage[subpreambles=true]{standalone}
\usepackage{import}

\usepackage{subfiles}
\usepackage[backend=biber,style=numeric,sorting=none]{biblatex}
\usepackage[pdfborder={0 0 0}]{hyperref}    % turns references into hyperlinks
\usepackage[margin=25mm]{geometry}  % adjusts page layout
\usepackage{graphicx}  % allows inclusion of PDF, PNG and JPG images
\usepackage[utf8]{inputenc}
\usepackage{textcomp}
\usepackage{import}
\usepackage[font=small,labelfont=bf]{caption}

\usepackage{amsmath}
\usepackage{booktabs}
\usepackage{textcomp}
\usepackage{csquotes}
\usepackage{color}

\usepackage{listings}

\newlength\gwidth
\newlength\gheight
\setlength{\gwidth}{\textwidth}
\setlength{\gwidth}{\textheight}

\definecolor{pblue}{rgb}{0.13,0.13,1}
\definecolor{pgreen}{rgb}{0,0.5,0}
\definecolor{pred}{rgb}{0.9,0,0}
\definecolor{pgrey}{rgb}{0.46,0.45,0.48}

\lstdefinestyle{HaskellStyle}{
  frame=none,
  xleftmargin=2pt,
  stepnumber=1,
  numbersep=5pt,
  numberstyle=\ttfamily\tiny\color[gray]{0.3},
  belowcaptionskip=\bigskipamount,
  commentstyle=\color{pgreen},
  keywordstyle=\color{pblue},
  stringstyle=\color{pred},
  basicstyle=\ttfamily,
  captionpos=b,
  escapeinside={&}{&},
  language=haskell,
  tabsize=2,
  emphstyle={\bf},
  commentstyle=\it,
  showspaces=false,
  columns=flexible,
  showstringspaces=false,
  morecomment=[l]
}

\lstnewenvironment{HaskellLst}{%
  \lstset{style=HaskellStyle}}{}

\usepackage{tikz}
\usepackage{forest}
\usepackage{calc}

\forestset{
  make tab/.style args={#1:#2:#3/#4:#5:#6/#7:#8:#9}{%
    content={%
      \tabcolsep=.6\tabcolsep
      \begin{tabular}{p{\widthof{x}}|p{\widthof{x}}|p{\widthof{x}}}
        #1 & #2 & #3\\\hline#4&#5&#6\\\hline#7&#8&#9
      \end{tabular}}},
  label position r/.initial=right,
  label position b/.initial=below
}

\usepackage[capitalize, nameinlink]{cleveref}
\crefdefaultlabelformat{#2\textbf{#1}#3}

\crefname{figure}{\textbf{Figure}}{\textbf{Figures}}
\crefname{table}{\textbf{Table}}{\textbf{Tables}}
\crefname{appendix}{\textbf{Appendix}}{\textbf{Appendices}}

\begin{document}
\section{Motivation}

Recent trends have seen functional approaches become increasingly dominant in the programming industry 
\cite{fp-trends}. This shift towards the functional languages has been driven by their numerous advantages over object oriented programming languages.  Primarily these are easily verifiable correctness properties, due the field's original derivation from Church's Lambda calculus.  Haskell is one such functional language that has risen to prominence following its first appearance in the 1990s. In addition to the generic
advantages offered by functional languages, Haskell’s ``pure, lazy'' nature makes it particularly attractive. ``Pure'' languages generate no side effects when functions are run, given the same input. 
Therefore every time a specific function is run in Haskell, an identical result is produced. ``Lazy'' languages evaluate expressions at the latest possible point, meaning if an expression is never required by a function, then the expression will never be reduced to a value. As such Haskell is an appealing choice for
projects that would benefit from the implementation of a functional language. However, Haskell is not widely used in industry at present. 

This project proposes to bridge this gap by implementing a Haskell to JVM compiler, which I have termed the Java Virtual-machine Haskell Compiler (JVHC). The JVM is a ubiquitous platform -- major software companies have already invested heavily into software that runs via the JVM. 
This is demonstrated by Java being the most popular programming language \cite{JVM-adoption}. 
As the JVHC allows Haskell code to be run on the JVM platform, it will significantly increase the accessibility and
longevity of code written in Haskell, while simultaneously increasing the flexibility available to the programmer 
when writing programs for the JVM. 
Furthermore, the JVHC allows Haskell code to take advantage of libraries already written for the JVM. 
As such this project will benefit the large sections of industry with established investments in the JVM. More generally, the JHVC will enable the industry to readily implement Haskell and reap the benefits of functional languages in new and existing projects.


% \begin{itemize}
%   \item 
%       Functional programming languages can be very useful in solving
%       problems, same for \textcolor{red}{imperative or 
%         object oriented languages}.
%       Having Haskell on the JVM will allow fusion between all 
%       libraries written in JVM based language and the dialect of Haskell
%       supported by JVHC.
%   \item 
%     Haskell is lazy, so can be useful in some problem domains to 
%     use a lazy language, gives more options to users of the JVM.
% \end{itemize}

\section{Overview of Haskell and its runtime}

The Haskell 98 \cite{haskell98-spec} report defines the Haskell language 
and a small standard library.
Haskell can be thought of as a lambda calculus, with an ML-like type system,
but with qualified types \cite{qualified-types}, known in 
Haskell as Type Classes. JVHC will omit some less crucial elements,
leave out parts of the standard library, 
and the compiler will  not support Type classes. While these limitations are trivial I do not feel they compromise the primary goal of the project  
(implementation of a lazy language (Haskell)
on the JVM with eager semantics). In future iterations of the compiler these limitations could be easily overcome, since the features for implementation are available via the JVM.
Some syntactic background is necessary to appreciate 
the benefits of Haskell in the context of software engineering.
Haskell allows definition of infinite size data structures:
\begin{HaskellLst}
nat :: Int -> [Int]
nat x = x : nat (x+1)
\end{HaskellLst}
This defines a function which when given an integer \verb|x|, will return 
a list of all the integers starting from \texttt{x}, and going on forever with
each element the successor of the previous element.
Since this list has no bound on the length it must be the case 
that values are computed only when required, 
this is possible due to the lazy nature of Haskell. 
The first line is an optional type signature, Haskell
uses \verb|::| to denote ``is of type'' and a single \verb|:| to denote cons (list append), unlike ML-like languages. Then \verb|[Int]| signifies a list
of \verb|Int| where \verb|Int| will be a fixed precision 
signed integer as defined in the Haskell 98 report 
\cite[\textsection6.4]{haskell98-spec}.
Then there is a function \verb|take :: Int -> [a] -> [a]| which will
take the first $n$ elements from a list returning a new list with at most
$n$ elements.
If then run 
\begin{HaskellLst}
take 5 (nat 2)
>> [2,3,4,5,6]
\end{HaskellLst}
then this will take the first 5 elements from the infinite list returning 
a finite list (once the expression is evaluated).

After defining some simple syntax, a more motivating example is presented. The example is 
inspired by the paper ``Why Functional Programming Matters''
\cite[\textsection5]{whyfpm}. 
In the following example a possibly infinite game tree for a simple
two player game can be defined 
\begin{HaskellLst}
data Tree a = Node a [Tree a]

repTree :: (Position -> [Position]) -> Position -> Node Position
repTree f x = Node x (map (repTree f) (f x))

gameTree :: Position -> Node Position
gameTree x = repTree moves x
\end{HaskellLst}
We assume that \verb|Position| is a type defined elsewhere in the program 
which stores a specific state in the game. 
\verb|moves| will be a function which returns all possible moves given a state.
Then calling \verb|gameTree startPos| will create
a game tree with every single possible move either player can make throughout
the entire game. The game could have infinitely many states, but since
Haskell is a lazy language the next state positions are only computed
when required. \cref{figure:tictactoeTree} shows the first two level of the game tree.
The value of each node in the game tree is only computed when required. Noughts and
crosses has a finite game tree, however this data structure can represent games 
with infinite states spaces. This example utilises both features
of Haskell laziness and algebraic datatypes (ADTs), the laziness allows an 
infinite datatype to be stored in memory and values will then only be computed when
required. ADTs allow concise definitions of the \verb|Tree| datatype.
\begin{figure}
  \centering
\begin{forest}
  TTT/.style args={#1:#2}{
    make tab/.expanded=\forestove{content},
    label={\pgfkeysvalueof{/forest/label position #1}:$#2$}
  },
  TTT*/.style={
    make tab=::/::/::,
    content/.expand once=%
    \expandafter\vphantom\expandafter{\romannumeral-`0\forestov{content}},
    draw=none,
    append after command={(\tikzlastnode.north) edge (\tikzlastnode.south)},
    for descendants={before computing xy={l*=1.2}},
  },
  th/.style=thick,
  for tree={node options=draw, inner sep=+0pt, parent anchor=south, child anchor=north}
%
[x::/::/::, TTT=r:\textcolor{pgrey}{\textsf{Player 1}}
 [x:o:/::/::, TTT=b:]
 [x::o/::/::, TTT=b:]
 [x::/o::/::, TTT=b:]
 [x::/:o:/::, TTT=b:]
 [x::/::o/::, TTT=b:]
 [x::/::/o::, TTT=b:]
 [x::/::/:o:, TTT=b:]
 [x::/::/::o, TTT=r:\textcolor{pblue}{\textsf{Player 2}}]
]
\end{forest}
\caption[Noughts and crosses game states]{First two levels of the set of all game states generated by the 
  \texttt{gameTree} function with the starting state at the top of 
  the tree signified by the label \textcolor{pgrey}{\textsf{Player 1}}.}
\label{figure:tictactoeTree}
\end{figure}
With care it is possible to define a breadth first walk of the tree, generating 
a list of each possible move at each state space. 

\begin{minipage}{\linewidth}
\begin{HaskellLst}
breadthFirst :: Tree a -> [a]
breadthFirst x = breadthFirst' [x]
  where
    breadthFirst' ((Node n lfs: ts)) =
      n : breadthFirst' (ts ++ lfs)
\end{HaskellLst}
\end{minipage}
This definition of an infinite length data structure is trivial in 
Haskell due to the lazy nature of the language -- implementing
a similar construct in the Java programming language would 
require the laziness to be explicitly specified, which 
would obfuscate the intent of the code. 

% \end{tikzpicture}

% \begin{itemize}
%   \item Explanation of simple Haskell features.

%   \item Define terms, lazy, pure, and functional.

%   \item Overview of writing idiomatic Haskell.

%   \item Talk about monads, since they are used throughout the 
%     implementation of the project.
  
%   \item Find an example of code that can be implemented more 
%     elegantly in Haskell then in Java.
% \end{itemize}

\section{Related work}

There are many implementations of the Haskell 98 report.
The de-facto standard implementation is GHC \cite{ghc} (Glasgow Haskell
Compiler), this compiler generates native machine code. There
are implementations that compile Haskell to JVM bytecode, including Eta \cite{eta} and Frege \cite{frege}. 
As other implementations exist, I can then 
compare my implementation to these compilers in the evaluation stage
of the project. 

One of the proposed extensions would be to implement an optimisation.,
I chose functional inlining as while its initial implementation is straightforward, 
there is significant scope to expand the quality of the optimisation. 
That said, numerous other optimisations would be possible. 
For example strictness analysis, and its associated then transformations. 


% \begin{itemize}
%   \item Other implementations of Haskell on the JVM include eta and
%     Frege. Then the baseline implementation of Haskell
%     is GHC, which the code for both Frege and eta is forked from (both
%     with their custom back-ends),.

%   \item 
%     Could talk about other possible optimisations, such as strictness
%     analysis, maybe compare to with inlining and justify why
%     inlining was a good choice.
% \end{itemize}

\end{document}
