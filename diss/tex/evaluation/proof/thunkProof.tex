\documentclass[float=false, crop=false]{standalone}

\usepackage[subpreambles=true]{standalone}
\usepackage{import}

\usepackage{verbatim}
\usepackage{amsmath}
\usepackage{amsfonts}
\usepackage{amssymb}
\usepackage{amsthm}
\usepackage{sansmath}
\usepackage{mathtools}

\usepackage{color}

\usepackage{lmodern}


\usepackage[utf8]{inputenc}
\usepackage[english]{babel}
 
\newtheorem{theorem}{Theorem}
\newtheorem{lemma}{Lemma}
\newtheorem{case}{Case}
\newtheorem{subcase}{Case}
\numberwithin{subcase}{case}

\newcommand{\tlang}{\bigstar}
\newcommand{\thunk}[1]{\lceil #1 \rceil}
\newcommand{\unwrap}[1]{\lfloor #1 \rfloor}
\newcommand{\tcbn}{\rightarrow_N}
\newcommand{\tcbv}{\rightarrow_V}
\newcommand{\tccbv}{\rightarrow_V^*}
\newcommand{\tthunk}{\rightarrow_\tlang}
\newcommand{\tlthunk}{\rightsquigarrow_\tlang}

\begin{document}
\section{Language definitions}

First define the core calculus 
\begin{align*}
  e ::=&\ c\ |\ x\ |\ \lambda x.e\ |\ e e'\
  \\ &|\ \mathtt{let}\ (x_1\ \mathtt{=}\ e_1,\dotsc, x_n\ \mathtt{=}\ e_n)\ \mathtt{in}\ e\
  \\ &|\ \mathtt{case}\ e\ \mathtt{of}\ b_1 \rightarrow e_1; \dots; b_n \rightarrow e_n
\end{align*} 
where 
\begin{itemize}
  \item $c\in\mathbb{Z}\cup\mathbb{CH}$ 
  \item $x\in \mathbb{L}$
  \item $e\in\mathbb{E}$ 
  \item $b$ is a possible case expression match called a binder.
  \item $\mathbb{L}$ is a set of labels.
  \item $\mathbb{CH}$ is the set of ASCII characters.
\end{itemize} 

Binders $b = (\mathit{TyCon},\{b^1,\dotsc,b^n\})$ is a $n$ size list of variables 
\mbox{$\{b^1,\dotsc,b^n \}\in \mathbb{L}^n$}. Tagged with its data type. If the expression
being considered is \mbox{\texttt{x :: xs -> e}} then this will be an arty 2 function
$\lambda x.\lambda \mathit{xs}. e$ with $b^1$ matching $x$ and $b^2$ matching $xs$.
The tag is either a custom data type constructor, a variable or a constant literal. 
Then values are defined as \[v_N ::= c\ |\ \lambda x.e \]
There is a finite map $\Sigma : \mathbb{L} 
\rightarrow \mathbb{E}$, that maps label to expressions. 
$\Sigma,x:e$ means add the pair $(x,e)$ to $\Sigma$ if $x \not\in\text{dom}(\Sigma)$
otherwise replace the current values in $\Sigma$ with the pair $(x,e)$.
There is also the following call-by-name transition relation 
\mbox{$\tcbn : \mathbb{E} \rightarrow \mathbb{E}$} is defined in Figure~\ref{t:cbn}. It makes sense to talk about one-step transition relations since
we are interested in proving a property about
a finite set of steps of execution.

\begin{figure}
\begin{align*}
  &\frac{}{\Sigma \vdash (\lambda x.e)e' \tcbn \Sigma,x:e' \vdash e } && \mathsf{Lam}_N \\ \\
  &\frac{\Sigma(x) = e}{\Sigma \vdash x \tcbn \Sigma \vdash e} && \mathsf{Var}_N\\ \\
  &\frac{\Sigma \vdash   e_1\tcbn \Sigma' \vdash e_1' \quad fv(e_2) \cap 
    (\Sigma' \setminus \Sigma) = \emptyset
  }{\Sigma \vdash e_1 e_2 \tcbn 
    \Sigma' \vdash e_1' e_2} && \mathsf{App}_N  \\ \\
  &\dfrac{}{\splitdfrac{\Sigma \vdash 
\mathtt{let}\ (x_1\ \mathtt{=}\ e_1,\dotsc, x_n\ \mathtt{=}\ e_n)\ \mathtt{in}\ e'}
    {\tcbn \Sigma,(x_1:e_1,\dotsc,x_n:e_n) \vdash e'}} && \mathsf{Let}_N  \\ \\
  &\dfrac{\Sigma \vdash e \rightarrow \Sigma' \vdash e'}{\splitdfrac{\Sigma \vdash 
  \mathtt{case}\ e\ \mathtt{of}\ b_1 \rightarrow e_1; \dots; b_n \rightarrow e_n}
  {\Sigma' \vdash\mathtt{case}\ e'\ \mathtt{of}\ b_1 \rightarrow e_1; \dots; b_n \rightarrow e_n}} 
      && \mathsf{Case\ 1}_N \\\\
  &\dfrac{b_i = \mathit{ctrType}(ty)}{\splitdfrac{\Sigma \vdash 
  \mathtt{case}\ (ty,\{a_1,\dotsc,a_m\}) \ \mathtt{of}\ b_1 \rightarrow e_1; \dots; b_n \rightarrow e_n}
  {\tcbn \Sigma,(b_i^1:a_1,\dotsc,b_i^m:a_m) \vdash e_i}} && \mathsf{Case\ 2}_N 
\end{align*}
\caption[The definition of the call-by-name semantics used by the core calculus]{The Call-By-Name semantics of the core calculus above.
  Here $b_i^j$ $j$th label in the $i$th binder in a case statement. And $\textit{ctrType}(ty) = b_i$ will
  find the smallest $b_i$ such that $ty_i = ty$ given $b_i = (ty_i,bs_i)$, 
  if the $\mathit{ty_i}$ is a variable then it will match any \textit{ty}.
   Note the condition on $\mathsf{App}_N$, which says that any label 
 expression pairs added to $\Sigma$ to make $\Sigma'$ must not
 also appear free in $e_2$, if those values do appear in $e_2$ then we can
 just $\alpha$-rename them.}
\label{t:cbn}
\end{figure}


Since I want to emulate these semantics using thunks, represented as $\thunk{e}$ in a call-by-value scheme, 
then thunks must be unwrapped, represented as $\unwrap{e}$. 
the new calculus will have the following syntax:
\begin{align*} 
  e_\tlang ::=&\ c\ |\ \lambda x.e_\tlang\ |\ e_\tlang e'_\tlang\ |\ \thunk{e_\tlang}\ |\ \unwrap{e_\tlang}
  \\ &|\ \mathtt{let}\ (x_1\ \mathtt{=}\ e_1,\dotsc, x_n\ \mathtt{=}\ e_n)\ \mathtt{in}\ e\
  \\ &|\ \mathtt{case}\ e\ \mathtt{of}\ b_1 \rightarrow e_1; \dots; b_n \rightarrow e_n
\end{align*}
with values 
\[ v_V ::= c\ |\ x\ |\ \thunk{e_\tlang}\ |\ \lambda x. e_\tlang \]
There is also another variable map $\Sigma_\tlang : \mathbb{L} \rightarrow \mathbb{E}_\tlang$, 
binders are the same as before.
There will be a translation relation $\tlthunk: \mathbb{E} \rightarrow \mathbb{E}_\tlang$
defined in Figure~\ref{t:translation}

\begin{figure}
\begin{align*}
  &\frac{}{c \tlthunk \thunk{c}} && \mathsf{Const}_\tlang\\ \\
  &\frac{}{x \tlthunk \thunk{x}} && \mathsf{Var}_\tlang\\ \\
  &\frac{e \tlthunk e_\tlang}
     {\lambda x . e \tlthunk \thunk{\lambda x_\tlang.\unwrap{e_\tlang}}}
       && \mathsf{Lam}_\tlang\\ \\
  &\frac{e  \tlthunk e_\tlang \quad e' \tlthunk e'_\tlang}
  {e e' \tlthunk \thunk{\unwrap{e_\tlang} e'_\tlang}} && \mathsf{App}_\tlang\\\\
  &\dfrac{e_1  \tlthunk e_{1_\tlang} \quad \dots \quad e_n \tlthunk e_{n_\tlang} \quad e' \tlthunk e_\tlang}
  {\splitdfrac{\mathtt{let}\ (x_1\ \mathtt{=}\ e_1,\dotsc, x_n\ \mathtt{=}\ e_n)\ \mathtt{in}\ e \tlthunk}
    {\thunk{\mathtt{let}\ 
        (x_1\ \mathtt{=}\ e_{1_\tlang},\dotsc, x_n\ \mathtt{=}\ e_{n_\tlang}) \ \mathtt{in}\ \unwrap{e_\tlang}}}} && \mathsf{Let}_\tlang\\ \\
  &\dfrac{e_1  \tlthunk e_{1_\tlang} \quad \dots \quad e_n \tlthunk e_{n_\tlang} \quad e' \tlthunk e'_\tlang}
  {\splitdfrac{\mathtt{case}\ e\ \mathtt{of}\ b_1 \rightarrow e_1; \dots; b_n \rightarrow e_n \tlthunk}
  {\thunk{\mathtt{case}\ \unwrap{e_\tlang}\ \mathtt{of}\ b_1 \rightarrow \unwrap{e_{1_\tlang}}; 
      \dots; b_n \rightarrow \unwrap{e_{n_\tlang}}}}} && \mathsf{Case}_\tlang
\end{align*}
\caption[Definition of translation relation from the core calculus to  calculus$_\tlang$]{Translation relation from the core calculus to the calculus$_\tlang$, also
extends the translation relation over expression of the form $\Sigma \vdash e \tlthunk \Sigma_\tlang 
\vdash e_\tlang$ by applying $\tlthunk$ for every expression in $\Sigma$ to make $\Sigma_\tlang$ and
then apply the relation as usual to $e$ to get $e_\tlang$.}
\label{t:translation}
\end{figure}

The call-by-value semantics for the calculus of expression $e_\tlang$ 
will be defined as in Figure~\ref{t:cbv}.

\begin{figure}
\begin{align*}
  &\frac{}{\Sigma_\tlang \vdash (\lambda x.e_\tlang)v \tcbv \Sigma_\tlang,x:v \vdash e_\tlang } && \mathsf{Lam}_V \\ \\
  &\frac{}{\Sigma_\tlang \vdash \unwrap{\thunk{e_\tlang}} \tcbv \Sigma_\tlang \vdash e_\tlang} 
  && \mathsf{Thunk}_V \\ \\
  &\frac{\Sigma_\tlang(x) = e_\tlang}{\Sigma_\tlang \vdash x \tcbv \Sigma_\tlang \vdash \unwrap{e_\tlang} } 
     && \mathsf{Var}_V\\ \\
  &\frac{\Sigma_\tlang \vdash   e_{1_\tlang}\tcbv \Sigma_\tlang' \vdash 
    e_{1_\tlang}' \quad fv(e_{2_\tlang}) 
    \cap (\Sigma'_\tlang \setminus \Sigma_\tlang) = 
    \emptyset }{\Sigma_\tlang \vdash e_{1_\tlang} e_{2_\tlang} \tcbv 
    \Sigma_\tlang' \vdash e_{1_\tlang}' e_{2_\tlang}} && \mathsf{App\ 1}_V\\\\
  &\frac{\Sigma_\tlang \vdash e_{2_\tlang} \tcbv \Sigma_\tlang'\vdash 
    e_{2_\tlang}}
  {\Sigma_\tlang \vdash v e_{2_\tlang} \tcbv 
    \Sigma_\tlang' \vdash v e_{2_\tlang}} && \mathsf{App\ 2}_V  \\ \\
  &\dfrac{}{\splitdfrac{\Sigma_\tlang \vdash 
\mathtt{let}\ (x_1\ \mathtt{=}\ e_{1_\tlang},\dotsc, x_n\ \mathtt{=}\
e_{n_\tlang})\ \mathtt{in}\ e_\tlang}
    {\tcbv \Sigma_\tlang,(x_1:e_{1_\tlang},\dotsc,x_n:e_{n_\tlang})
      \vdash e_\tlang}} && \mathsf{Let}_V  \\ \\
  &\dfrac{\Sigma_\tlang \vdash e_\tlang \tcbv \Sigma_\tlang' \vdash e'_\tlang}
  {\splitdfrac{\Sigma_\tlang \vdash 
  \mathtt{case}\ e_\tlang\ \mathtt{of}\ b_1 \rightarrow e_{1_\tlang}; 
  \dots; b_n \rightarrow e_{n_\tlang}}
  {\tcbv \Sigma_\tlang' \vdash\mathtt{case}\ e_\tlang'\ 
    \mathtt{of}\ b_1 \rightarrow e_{1_\tlang}; 
    \dots; b_n \rightarrow e_{n_\tlang}}} 
      && \mathsf{Case\ 1}_V \\\\
  &\dfrac{ty = \mathit{ctrType}(b_i)}{\splitdfrac{\Sigma_\tlang \vdash 
  \mathtt{case}\ (ty,\{a_{1_\tlang},\dotsc,a_{m_\tlang}\})
  \ \mathtt{of}\ b_1 \rightarrow e_{1_\tlang};
  \dots; b_n \rightarrow e_{n_\tlang}}
  {\tcbv \Sigma_\tlang,(b_i^1:a_{1_\tlang},\dotsc,b_i^m:a_{m_\tlang})
    \vdash e_{i_\tlang}}} && \mathsf{Case\ 2}_V 
\end{align*}
\caption[The call-by-value semantics of calculus$_\tlang$]{The call-by-value semantics of the core calculus above. In the 
  $\mathsf{App\ 2}_V$ rule $v$ is any value defined by the grammar $v_V$.}
\label{t:cbv}
\end{figure}



Define $\tccbv: \mathbb{E}_\tlang \rightarrow \mathbb{E}_\tlang$ as the transitive closure of $\tcbv$
\begin{align*}
  &\frac{}{e_\tlang \tccbv e_\tlang} && \mathsf{Refl}\\ \\
  &\frac{e_\tlang \tcbv e'_\tlang}{e_\tlang \tccbv e'_\tlang} 
    && \mathsf{Closure}\\ \\
  &\frac{e_\tlang \tccbv e'_\tlang \quad e'_\tlang \tccbv e_\tlang'' }{e_\tlang \tccbv e''_\tlang} && \mathsf{Trans}
\end{align*}

\section{Proofs}

For the purpose of this proof I will use the term well typed to mean a program in which the 
type can be inferred, but will not formally define the notion of being well-typed.

\begin{theorem}[Evaluation equivalence]
If the expression $e$ is well typed and
\mbox{$\Sigma \vdash e \tcbn \Sigma' \vdash e'$}, $\Sigma \tlthunk \Sigma_\tlang$ and \mbox{$e \tlthunk e_\tlang$} 
then for some $a_\tlang \in \mathbb{E}_\tlang$ 
\mbox{$\Sigma_\tlang \vdash \unwrap{e_\tlang}\tccbv \Sigma'_\tlang \vdash \unwrap{e_\tlang'}$},
\mbox{$\Sigma' \tlthunk \Sigma'_\tlang$}, \mbox{$e' \tlthunk a_\tlang$} where 
\mbox{$\Sigma'_\tlang \vdash a_\tlang \tccbv \Sigma'_\tlang \vdash  e'_\tlang$}.
\end{theorem}



This will be proved by rule induction over the $\tcbn$ relation. 

\begin{proof}
  \begin{case}[$\mathbf{Lam}_N$]
    This is the form of the lambda reduction rule under call-by-name semantics.
    \[\frac{}{\Sigma \vdash (\lambda x.a)a' \tcbn \Sigma,(x:a') \vdash a }
        \quad \mathsf{Lam}_N\]
    Then \[ \frac{\Sigma \tlthunk \Sigma' \quad 
        a \tlthunk a_\tlang \quad a' \tlthunk a'_\tlang}
      {\Sigma \vdash (\lambda x .a)a' 
        \tlthunk \Sigma \vdash \thunk{\unwrap{
            \thunk{\lambda x. \unwrap{ a_\tlang}}} a'_\tlang} }
              \quad \mathsf{Lam}_\tlang\]
    And \[ \frac{\Sigma \tlthunk \Sigma_\tlang \quad a 
        \tlthunk a_\tlang \quad a' \tlthunk a'_\tlang}
      {\Sigma,(x,a') \vdash a\tlthunk \Sigma_\tlang,
        (x,a'_\tlang) \vdash a_\tlang} \quad \mathsf{IH}_\tlang \]
   Then looking at the reduction set of  
   ${\unwrap{\thunk{\unwrap{
        \thunk{\lambda x. \unwrap{ a_\tlang}}} a'_\tlang}}}$
    \begin{align*}
     &\frac{}{ 
       \Sigma_\tlang \vdash \unwrap{\thunk{\unwrap{\thunk{\lambda x.
               \unwrap{ a_\tlang}}} a'_\tlang}} \tcbv 
       \Sigma_\tlang \unwrap{\thunk{\lambda x.
               \unwrap{ a_\tlang}}} a'_\tlang} && \mathsf{Thunk}_V \\ \\
     &\cfrac{\cfrac{}{\Sigma_\tlang \vdash \unwrap{\thunk{\lambda x.
           \unwrap{ a_\tlang}}} \tcbv  \Sigma_\tlang \vdash \lambda x.
       \unwrap{a_\tlang}}}{ 
       \Sigma_\tlang \vdash \unwrap{\thunk{\lambda x.
               \unwrap{ a_\tlang} a'_\tlang}} \tcbv 
           \Sigma_\tlang \vdash (\lambda x.\unwrap{a_\tlang}) a'_\tlang} && 
         \frac{\mathsf{Thunk}_V}{\mathsf{App\ 1}_V}\\ \\
     &\frac{}{\Sigma_\tlang \vdash (\lambda x.\unwrap{a_\tlang})a'_\tlang \tcbv 
       \Sigma_\tlang, (x:a_\tlang') \vdash \unwrap{a_\tlang}} && \mathsf{Lam}_V
    \end{align*}
    This case then holds.
  \end{case}

  \begin{case}[$\mathsf{App}_N$]
    This is the rule for $\textsf{App}_N$ is of the form
    \begin{align*}
      \frac{\Sigma \vdash a \tcbn \Sigma' \vdash a'}
      {\Sigma \vdash ab \tcbn \Sigma' \vdash a' b} && \mathsf{App}_N
    \end{align*}
    Then $e$ is of the form $e = a b$, looking at the $\tlthunk$ relation.
    \[\frac{\Sigma \tlthunk \Sigma_\tlang \quad a 
        \tlthunk a_\tlang \quad b \tlthunk a_\tlang}
      { \Sigma \vdash a b\tlthunk \Sigma_\tlang \vdash 
        \thunk{\unwrap{a_\tlang }b_\tlang}} \quad \mathsf{App}_\tlang\]
    and applying the reduction relation to the result of the translation 
    relation gives
    \begin{align*}
      &\frac{}{\Sigma_\tlang \vdash \unwrap{\thunk{\unwrap{a_\tlang} b_\tlang} }  
        \tcbv \Sigma_\tlang  \unwrap{a_\tlang} b_\tlang} && \mathsf{Thunk}_V\\ \\
      &\frac{\Sigma_\tlang \vdash \unwrap{a_\tlang} \tccbv \Sigma'_\tlang \vdash \unwrap{a'_\tlang}}{
      \Sigma_\tlang \vdash \unwrap{a_\tlang} b_\tlang \tccbv \Sigma'_\tlang \vdash 
      \unwrap{a'_\tlang} b_\tlang } &&
      \mathsf{App}_V
    \end{align*}
    then it can be concluded that 
    \[ \frac{}{\Sigma_\tlang \vdash \unwrap{\thunk{\unwrap{a_\tlang} b_\tlang}}
        \tccbv \Sigma_\tlang \unwrap{a'_\tlang} b_\tlang  } \]
    Finally \[ \frac{\Sigma' \tlthunk \Sigma'_\tlang \quad a' \tlthunk a'_\tlang \quad b \tlthunk b_\tlang}
      {\Sigma' \vdash a' b \tlthunk \Sigma'_\tlang \vdash \thunk{\unwrap{a'_\tlang} b_\tlang} } \quad \mathsf{App}_\tlang \]
    Which will reduce to
    \[\Sigma'_\tlang \vdash \unwrap{\thunk{\unwrap{a'_\tlang}b_\tlang}} \tcbv \Sigma'_\tlang \vdash
    \unwrap {a'_\tlang} b_\tlang \quad \mathsf{Thunk}_V\]

  \end{case}

  \begin{case}[$\mathsf{Var}_N$]
    \[ \frac{\Sigma(x) = e}{\Sigma \vdash x \tcbn \Sigma \vdash e} \quad
      \mathsf{Var}_N \]
    The translation relation says that \[ \frac{}{x \tlthunk \thunk{x}} 
      \quad \mathsf{Var}_\tlang \]
    Then 
    \begin{align*}
    & \Sigma_\tlang \vdash \unwrap{\thunk{x}} \tcbv \Sigma_\tlang \vdash x  && \mathsf{Thunk}_V \\ \\
    & \frac{\Sigma_\tlang(x) = e_\tlang}
    {\Sigma_\tlang \vdash x  \tcbv \Sigma_\tlang \vdash \unwrap{e_\tlang}} && \mathsf{Var}_V 
  \end{align*}
    This is the same as produced by the translation relation 
    \mbox{$e \tlthunk e_\tlang$}
  \end{case}

\begin{case}[$\mathsf{Let}_N$]
    This is the form of the rule:
    \[
  \dfrac{}{\splitdfrac{\Sigma \vdash \mathtt{let}\ 
      (x_1\ \mathtt{=}\ a_1,\dotsc, x_n\ \mathtt{=}\ a_n)\ \mathtt{in}\ e'}
    {\tcbn \Sigma,(x_1:a_1,\dotsc,x_n:a_n) \vdash e'}} \quad \mathsf{Case}_N\]

  Then the expression is of the form above 
  $e = \mathtt{let}\ (x_1\ \mathtt{=}\ a_1,\dotsc, x_n\ \mathtt{=}\ a_n)\ \mathtt{in}\ e$
  
\[\dfrac{\Sigma \tlthunk \Sigma_\tlang \quad a_1  \tlthunk a_{1_\tlang} 
    \quad \dots \quad a_n \tlthunk a_{n_\tlang} \quad a' \tlthunk a_\tlang}
  {\splitdfrac{\Sigma \vdash \mathtt{let}\ (x_1\ \mathtt{=}\ a_1,\dotsc, 
      x_n\ \mathtt{=}\ a_n)\ \mathtt{in}\ a \tlthunk}
    {\Sigma_\tlang \vdash \thunk{\mathtt{let}\ 
        (x_1\ \mathtt{=}\ a_{1_\tlang},\dotsc, x_n\ \mathtt{=}\ a_{n_\tlang})
        \ \mathtt{in}\ \unwrap{a_\tlang}}}} \quad \mathsf{Case}_\tlang\]

\begin{align*} 
  & \Sigma_\tlang \vdash \unwrap{\thunk{\mathtt{let}\ 
        (x_1\ \mathtt{=}\ a_{1_\tlang},\dotsc, x_n\ \mathtt{=}\ a_{n_\tlang})
        \ \mathtt{in}\ \unwrap{a_\tlang}}}\\ \tcbv &\Sigma_\tlang \vdash
    \mathtt{let}\ (x_1\ \mathtt{=}\ a_{1_\tlang},
    \dotsc, x_n\ \mathtt{=}\ a_{n_\tlang})
    \ \mathtt{in}\ \unwrap{a_\tlang} && \mathsf{Thunk}_V \\ \\
  & \Sigma_\tlang \vdash \mathtt{let}\ 
    (x_1\ \mathtt{=}\ a_{1_\tlang},\dotsc, x_n\ \mathtt{=}\ a_{n_\tlang})
    \ \mathtt{in}\ \unwrap{a_\tlang}\\ \tcbv &\Sigma_\tlang,
    (x_1:a_{1_\tlang},\dotsc,x_n:a_{n_\tlang}) \vdash 
    \unwrap{a_\tlang} && \mathsf{Let}_V
\end{align*}

Then also note that 
\[
  \dfrac{\Sigma \tlthunk \Sigma_\tlang \quad a_1  
    \tlthunk a_{1_\tlang} \quad \dots \quad a_n 
    \tlthunk a_{n_\tlang} \quad a' \tlthunk a_\tlang}
  {\splitdfrac{\Sigma,(x_1:a_1,\dotsc,x_n:a_n) \vdash a \tlthunk}
   {\Sigma_\tlang,(x_1:a_{1_\tlang},\dotsc,x_n:a_{n_\tlang}) \vdash a_\tlang} }
    \quad \mathsf{IH}_\tlang\] From the induction hypothesis.
    Hence done.
\end{case}

\begin{case}[$\mathsf{Case\ 1}_N$]
Similar to $\mathsf{App\ 1}_N$
\end{case}

\begin{case}[$\mathsf{Case\ 2}_N$]
  Note the form of the transition rule:
  \[ \dfrac{ty = \mathit{ctrType}(b_i)}{\splitdfrac{\Sigma \vdash 
  \mathtt{case}\ (ty,\{a_1,\dotsc,a_m\}) \ \mathtt{of}\ b_1 \rightarrow c_1; \dots; b_n \rightarrow c_n}
  {\tcbn \Sigma,(b_i^1:a_1,\dotsc,b_i^m:a_m) \vdash c_i}} \]
  We have $e = 
  \mathtt{case}\ (ty,\{a_1,\dotsc,a_m\}) \ \mathtt{of}\ b_1 \rightarrow c_1;
  \dots; b_n \rightarrow c_n$ and
  $e' = c_i$
  Consider the translation relation
  \[ \dfrac{c_1  \tlthunk c_{1_\tlang} \quad \dots \quad c_n \tlthunk c_{n_\tlang} 
      \quad c' \tlthunk c'_\tlang}
  {\splitdfrac{\mathtt{case}\ (ty,\{a_1,\dotsc,a_n\})\ \mathtt{of}\ 
      b_1 \rightarrow c_1; \dots; b_n \rightarrow c_n \tlthunk}
  {\thunk{\mathtt{case}\ c_\tlang\ \mathtt{of}\ b_1 \rightarrow \unwrap{c_{1_\tlang}}; 
      \dots; b_n \rightarrow \unwrap{c_{n_\tlang}}}}} \]
  for $\unwrap{e}$
  \begin{align*} 
  &\Sigma_\tlang \vdash \unwrap{\thunk{\mathtt{case}\ c_\tlang\ 
      \mathtt{of}\ b_1 \rightarrow \unwrap{c_{1_\tlang}}; 
      \dots; b_n \rightarrow \unwrap{c_{n_\tlang}}}}\\ \tcbv 
  &\Sigma_\tlang \vdash
  \mathtt{case}\ c_\tlang\ \mathtt{of}\ b_1 \rightarrow \unwrap{c_{1_\tlang}};  
      \dots; b_n \rightarrow \unwrap{c_{n_\tlang}} && \mathsf{Thunk}_V\\ \\
 & \Sigma_\tlang \vdash  \mathtt{case}\ c_\tlang\ \mathtt{of}\ b_1 \rightarrow \unwrap{c_{1_\tlang}};  
      \dots; b_n \rightarrow \unwrap{c_{n_\tlang}}\\ \tcbv
      &\Sigma_\tlang,(x_1:a_{1_\tlang},\dotsc,x_n:a_{n_\tlang}) \vdash \unwrap{c_{i_\tlang}}
      &&\mathsf{Case\ 2}_V\\ \\
\end{align*}
  Note that from the induction hypothesis
  \[ \frac{\Sigma \tlthunk \Sigma_\tlang \quad c_1 \tlthunk c_{1_\tlang} \quad \dots \quad
    c_n \tlthunk c_{n_\tlang}}{\Sigma (x_1:a_1,\dotsc,x_n:a_n) \vdash c_i \tlthunk \Sigma_\tlang \vdash 
    (x_1:a_{1_\tlang},\dotsc,x_n:a_{n_\tlang}) \vdash c_{i_\tlang}} \quad 
  \mathsf{IH}_\tlang\]
Hence this case holds.
\end{case}

\end{proof}



\end{document}
