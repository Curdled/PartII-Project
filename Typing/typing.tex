\documentclass[12pt,a4paper,twoside]{article}
\usepackage[pdfborder={0 0 0}]{hyperref}
\usepackage[margin=25mm]{geometry}
\usepackage{graphicx}
\usepackage{lmodern}
\usepackage{textcomp}
\usepackage[T1]{fontenc}
\usepackage{parskip}
\usepackage{verbatim}
\usepackage{amsfonts}
\usepackage{amsmath}
\usepackage{xcolor}
\usepackage[normalem]{ulem}
\def\comment#1\done{{\color{blue}#1}}
\def\add#1\done{{\color{red}#1}}
\def\remove#1\done{{\sout{#1}}}
\def\change#1\to#2\done{{\sout{#1}\color{red}#2}}

\usepackage[backend=biber,style=numeric,sorting=none]{biblatex}

\bibliography{refs.bib}

\begin{document}


\begin{center}
  \Large
  Typing Core\\[4mm]

  J.~Isaacs
  \today
\end{center}

\section{Representation}

\subsection{Kind}

Haskell has the notion of \texttt{Kind}s, I will be supporting:

\begin{align*}
K =& *\\ 
  &|\ K \rightarrow K
\end{align*}

where a types like \texttt{Int} and \texttt{Char} will have a kind \texttt{*}, but type constructors like \texttt{Maybe} and 
\texttt{IO} will have kind \texttt{* $\rightarrow$ *}.

I will represent kinds using the following type:

\begin{verbatim}
data Kind = Star | KArrow Kind Kind
\end{verbatim}

As defined in \cite{jones1999typing}

\subsection{Types}

Haskell type expressions can either be type variables constants or applications of one type to another
Type can be defined using:

\begin{verbatim}
data Type = TVar Id   Kind
          | TCon Id   Kind
          | TApp Type Type
          | TGen Int
\end{verbatim}

Then primitave types can be defined as:

\begin{verbatim}
tInt   = TCon "Int" Star
tArrow = TCon "->"  (KArrow Star (KArrow Star Star))
\end{verbatim}

Then the type $a \rightarrow b$ will be represented as:

\begin{verbatim}
TApp (TApp tArrow 
           (TVar "a" Star)) 
     (TVar "b" Star)
\end{verbatim}

\subsection{Expressions}

\begin{verbatim}
data Expr = Var  Var
          | Lit  Literal
          | App  Expr    Expr
          | TApp Expr    Expr
          | Lam  Var     Expr
          | BLam Var     Expr
          | Let  Bind    Expr
          | Case Expr    Var  Type [(AltCon,[Var],Expr)]
          | Type Type
\end{verbatim}

We will first convert from the parsed AST into a System F inspired 
core language without $\Lambda$ then infer the types of each \texttt{Expr}
and then finally once we know the type add the type specialization and
type abstractions (i.e. $\Lambda$).

From \texttt{u = \textbackslash x -> v x} and \texttt{v = \textbackslash y -> y}:
\begin{verbatim}
TFunDecl (TVarPat (TVarID "u") []) (TELambda "x" (TEApp (TEVar "v") (TEVar "x")))
TFunDecl (TVarPat (TVarID "v") []) (TELambda "y" (TEVar "y"))
\end{verbatim}
which would desugar to:
\begin{verbatim}
Lam "x" (App (Var "v") (Var "x"))
Lam "y" (Var "y")
\end{verbatim}
Then I would inter the types of \texttt{u} and \texttt{v} to be:
\begin{verbatim}
u :: forall a. a->a
v :: forall a. a->a
\end{verbatim}

We first check to see if the inferred types have \texttt{forall} types in which
case we must add a $\Lambda$ to be term abstracting over that type.

Then I can rewrite \texttt{u} and \texttt{v} to be bound to:
\begin{verbatim}
u => BLam "a" (Lam "x" (App 
                        (TApp (Var "v") (Var "a")) 
                        (Var "x")))
v => BLam "a" (Lam "y" (Var "y"))
\end{verbatim}

This will require checking that on an application if the function that the 
argument is being applied to has a $\Lambda$ in front meaning we must first 
apply a type to this function then the term.


\printbibliography

\end{document}
