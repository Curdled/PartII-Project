\documentclass[12pt,a4paper,twoside]{article}
\usepackage[pdfborder={0 0 0}]{hyperref}
\usepackage[margin=25mm]{geometry}
\usepackage{graphicx}
\usepackage{lmodern}
\usepackage{textcomp}
\usepackage[T1]{fontenc}
\usepackage{parskip}
\usepackage{verbatim}
\usepackage{amsfonts}
\usepackage{amsmath}
\usepackage{xcolor}
\usepackage[normalem]{ulem}
\def\comment#1\done{{\color{blue}#1}}
\def\add#1\done{{\color{red}#1}}
\def\remove#1\done{{\sout{#1}}}
\def\change#1\to#2\done{{\sout{#1}\color{red}#2}}


\begin{document}


\begin{center}
  \Large
  Haskell Informal Semantics\\[4mm]
  \today
\end{center}

\section{Preface}

This document was made from the Haskell 98 Report, which was modified to specify the language this compiler will accept and produce executable programs for.

\section{Lambda}

I don't allow currying (more than a single variable in the lambda expression) or deconstructing data constructors in the lambda expression

\section{Pattern Matching}

Pattern matching can either fail; succeed (returning a binding for each variable in the pattern); or diverge. Pattern matching proceeds from left to right and top to bottom

\begin{itemize}
  \item[1] Matching the pattern var against a value v always succeeds and binds var to v.

  \item[4] Matching the pattern \textit{con var1...varn} against a value, where \texttt{con} is a constructor defined by data, depends on the value:
    \begin{itemize}
      \item If the value is of the form \texttt{con v1...vn} then the match succeeds.
      \item If the value fo teh form \texttt{con' v1...vn} where $\texttt{con'} \neq \texttt{con}$ then the match fails.
      \item If the value is \_|\_ then the match will diverge
    \end{itemize}
    I only allow a single \texttt{con} is each pattern of a case statement hence \texttt{con pat1...patn} goes to \texttt{con var1...varn}

  \item[7] Matching a numeric, char or string pattern $k$ against a value $v$ literal succeeds if $k==v$ where $==$ will be defined in the compiler.

    There are not type classes hence the next best thing is to define a $==$ for int, char and string since these are the only supported types
\end{itemize}

    Modified version of informal semantics of pattern matching from 3.17.2 of the Haskell report. Only the applicable rules are listed.

\section{Algebraic Datatype Declarations (ADTs)}

The compiler only supports the \texttt{data} declaration, this is of the form

\[\texttt{data}\ T u_1...u_k = K_1 t_{11}...t_{ki} | ...| K_n t_{n1} ... t_{nj}\]

This will introduce a new type constructor $T$ with one or more data constructors $K_1,...,K_n$.

The type of a data constructor will be 
\[K_i :: \texttt{forall}\ u_1...u_k.\ t_{i1} \rightarrow ... \rightarrow t_{ij} \rightarrow T(u_1...u_k) \]

These will be an interface \[ \texttt{interface T<$u_1,...,u_k$>}\] in Java with $n$ subclasses where $K_i$ will be a class 

\[ \texttt{class $K_i$ implements interface T<$u_1,...,u_k$>} \] with a constructor \[K_i (t_{i1}\ \ ti1,...,t_{ij}\ \ tij)\{...\}\] with methods to allow
the currying of data constructor arguments.

\section{4.4.1 Type Signatures}

Type signatures have the form $v_1,...,v_n :: t$, which is a equivalent to $\forall i. v_i :: t$ where $v_i$ where $i$ goes from 1 to $n$. 
Each $v_i$ must have a value binding in the same declaration list.
One variable cannot have more than a single type signature. $v_1$ is a variable and $t$ is the value for the signature. 

\section{4.5.2 Type Generalization}

The Hindley-Milner type system assigns types to a let-expression in two stages. First, the right-hand side of the declaration is typed, 
giving a type with no universal quantification. Second, all type variables that occur in this type are universally 
quantified unless they are associated with bound variables in the type environment; this is called generalization. Finally, the body of the let-expression is typed.

\section{Named functions}

Named functions (e.g. \texttt{f x y = plus x y}) will be written to (\texttt{f \textbackslash x -> \textbackslash y -> plus x y}), so they can be treated uniform

\section{4.5.\{4,5\}}

Sometimes it is not possible to generalize over all the type variables used in the type of the definition. For example, consider the declaration 

\begin{verbatim}
  f x = let g y z = ([x,y], z)
        in ...
\end{verbatim}

In an environment where \texttt{x} has type a, the type of \texttt{g}'s definition is \texttt{a ->b ->([a],b)}. 
The generalization step attributes to g the type \texttt{forall b.a->b->([a],b)}; 
only \texttt{b} can be universally quantified because \texttt{a} occurs in the type environment. 
We say that the type of \texttt{g} is monomorphic in the type variable \texttt{a}.
The effect of such monomorphism is that the first argument of all applications of \texttt{g} must be of a single type. For example, it would be valid for the `...' to be 

  \[ \texttt{(g True, g False)} \]

(which would, incidentally, force \texttt{x} to have type \texttt{Bool}) but invalid for it to be 

  \[ \texttt{(g True, g 'c')} \]

In general, a type \texttt{forall u.t} is said to be monomorphic in the type variable \texttt{a} if \texttt{a} is free in \texttt{forall u.t}.

It is worth noting that the explicit type signatures provided by Haskell are not powerful enough to express types that include monomorphic type variables. For example, we cannot write 

\begin{verbatim}
  f x = let 
          g :: a -> b -> ([a],b)
          g y z = ([x,y], z)
        in ...
\end{verbatim}

because that would claim that \texttt{g} was polymorphic in both \texttt{a} and \texttt{b} (Section 4.4.1). In this program, \texttt{g} can only be given a type signature if its first argument is restricted to a type not involving type variables; for example 


 \[\texttt{ g :: Int -> b -> ([Int],b)}\]

This signature would also cause x to have type Int.

\section{4.6 Kind Inference}

This be the same as in the report

\end{document}
